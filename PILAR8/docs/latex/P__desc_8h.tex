\section{P\_\-desc.h File Reference}
\label{P__desc_8h}\index{P_desc.h@{P\_\-desc.h}}
Interface module for protein descriptors. 


\subsection*{Functions}
\begin{CompactItemize}
\item 
int {\bf read\_\-feature} (char $\ast$file, {\bf feature\_\-t} $\ast$f)
\begin{CompactList}\small\item\em Reads data values from the especified feature values file.\item\end{CompactList}\item 
{\bf status} {\bf get\_\-feature} ({\bf feature\_\-t} $\ast$f)
\begin{CompactList}\small\item\em Prompt user for feature file and read it.\item\end{CompactList}\item 
{\bf status} {\bf feature\_\-name} (char $\ast$file)
\begin{CompactList}\small\item\em Returns the name of the last feature read.\item\end{CompactList}\item 
{\bf status} {\bf read\_\-descriptor} (char $\ast$file, {\bf descriptor\_\-t} $\ast${\bf d})
\begin{CompactList}\small\item\em Read descriptor.\item\end{CompactList}\end{CompactItemize}


\subsection{Detailed Description}
Interface module for protein descriptors.

 

 Header file with the definitions pertaining to module {\bf P\_\-descr.c}. It is usually included via {\bf P\_\-extern.h}

\begin{Desc}
\item[Precondition: ]\par
Requires PORTABLE.H for definition of type status.\end{Desc}
\begin{Desc}
\item[See also: ]\par
{\bf portable.h}\end{Desc}
\begin{Desc}
\item[Author: ]\par
J. R. Valverde  \end{Desc}
\begin{Desc}
\item[Date: ]\par
8-apr-1990\end{Desc}


\subsection{Function Documentation}
\index{P_desc.h@{P\_\-desc.h}!feature_name@{feature\_\-name}}
\index{feature_name@{feature\_\-name}!P_desc.h@{P\_\-desc.h}}
\subsubsection{\setlength{\rightskip}{0pt plus 5cm}{\bf status} feature\_\-name (char $\ast$ {\em file})}\label{P__desc_8h_a2}


Returns the name of the last feature read.



 \begin{Desc}
\item[Parameters: ]\par
\begin{description}
\item[{\em 
file}]String to store the name.\end{description}
\end{Desc}
\begin{Desc}
\item[Returns: ]\par
SUCCESS if so, FAIL if there is not current feature name.\end{Desc}
\begin{Desc}
\item[Warning: ]\par
This function is HIGHLY SYSTEM DEPENDENT. Supposes that f\_\-name is long enough to hold the name. \end{Desc}
\index{P_desc.h@{P\_\-desc.h}!get_feature@{get\_\-feature}}
\index{get_feature@{get\_\-feature}!P_desc.h@{P\_\-desc.h}}
\subsubsection{\setlength{\rightskip}{0pt plus 5cm}{\bf status} get\_\-feature ({\bf feature\_\-t} $\ast$ {\em f})}\label{P__desc_8h_a1}


Prompt user for feature file and read it.



 Asks the user for the feature to be used in the search and reads it from the file.\begin{Desc}
\item[Parameters: ]\par
\begin{description}
\item[{\em 
f}]Table to read data values into.\end{description}
\end{Desc}
\begin{Desc}
\item[Returns: ]\par
SUCCESS if read's been OK. FAIL otherwise. \end{Desc}
\index{P_desc.h@{P\_\-desc.h}!read_descriptor@{read\_\-descriptor}}
\index{read_descriptor@{read\_\-descriptor}!P_desc.h@{P\_\-desc.h}}
\subsubsection{\setlength{\rightskip}{0pt plus 5cm}{\bf status} read\_\-descriptor (char $\ast$ {\em file}, {\bf descriptor\_\-t} $\ast$ {\em d})}\label{P__desc_8h_a3}


Read descriptor.



 \begin{Desc}
\item[Parameters: ]\par
\begin{description}
\item[{\em 
file}]Name of file containing the descriptor definition. \item[{\em 
d}]Structure to store the descriptor in.\end{description}
\end{Desc}
\begin{Desc}
\item[Returns: ]\par
TRUE if all went well, FALSE otherwise. \end{Desc}
\index{P_desc.h@{P\_\-desc.h}!read_feature@{read\_\-feature}}
\index{read_feature@{read\_\-feature}!P_desc.h@{P\_\-desc.h}}
\subsubsection{\setlength{\rightskip}{0pt plus 5cm}int read\_\-feature (char $\ast$ {\em file}, {\bf feature\_\-t} $\ast$ {\em f})}\label{P__desc_8h_a0}


Reads data values from the especified feature values file.



 \begin{Desc}
\item[Parameters: ]\par
\begin{description}
\item[{\em 
file}]Name of file containing values. \item[{\em 
f}]Feature table filled with the values.\end{description}
\end{Desc}
\begin{Desc}
\item[Note: ]\par
Values must be stored in the following format: \char`\"{}Amino\-Acid (1 letter code) HTAB Value CR/LF\char`\"{}\end{Desc}
\begin{Desc}
\item[Returns: ]\par
Number of amino acid values read. \end{Desc}
