\section{P\_\-bits.c File Reference}
\label{P__bits_8c}\index{P_bits.c@{P\_\-bits.c}}
Bit manipulation routines. 


{\tt \#include $<$stdio.h$>$}\par
\subsection*{Compounds}
\begin{CompactItemize}
\item 
struct {\bf \_\-bv}
\begin{CompactList}\small\item\em Bit vector structure.\item\end{CompactList}\end{CompactItemize}
\subsection*{Typedefs}
\begin{CompactItemize}
\item 
typedef {\bf \_\-bv} $\ast$ {\bf bit\_\-vector}
\end{CompactItemize}
\subsection*{Functions}
\begin{CompactItemize}
\item 
{\bf bit\_\-vector} {\bf bv\_\-new} (bits) unsigned bits
\begin{CompactList}\small\item\em bit\_\-vector {\bf bv\_\-new}(bits) {\rm (p.\,\pageref{P__bits_8c_a6})} Create a new bit vector and set to zeros Reserve the memory space needed for the bit vector.\item\end{CompactList}\end{CompactItemize}
\subsection*{Variables}
\begin{CompactItemize}
\item 
char {\bf value}
\item 
unsigned {\bf bitnum}
\item 
{\bf \_\-bv} $\ast$ {\bf bv2}
\item 
{\bf \_\-bv} $\ast$ {\bf bv3}
\item 
{\bf \_\-bv} $\ast$ {\bf bv}
\end{CompactItemize}


\subsection{Detailed Description}
Bit manipulation routines.

 

 M\'{o}dulo con rutinas de manipulaci\'{o}n de vectores de bits extra\'{\i}do de una librer\'{\i}a de rutinas para CP/M

CONTENIDO: bv\_\-new(bv, bits) bv\_\-fill(bv, value) bv\_\-set(bv, bitnum) bv\_\-test(bv, bitnum) bv\_\-nz(bv) bv\_\-and(bv3, bv1, bv2) bv\_\-or(bv3, bv1, bv2) bv\_\-xor(bv3, bv1, bv2) bv\_\-disp(title, bv)

\begin{Desc}
\item[Precondition: ]\par
stdio.h \par
alloc.h \par
mem.h\end{Desc}
ULTIMA MODIFICACION: LIBRARY.C, LIBRARY.H by A. J.-L. (1983) Traducci\'{o}n al castellano (1986) Extracci\'{o}n e implementaci\'{o}n en PC-AT (Jos\'{e} Ram\'{o}n Valverde Carrillo, 1989)

\begin{Desc}
\item[Author: ]\par
Andy Johnson-Laird. The programmer's CP/M handbook. Ed. Osborne / Mc\-Graw-Hill. 1983\end{Desc}
COPYRIGHT: Mc\-Graw-Hill (1983)



\subsection{Typedef Documentation}
\index{P_bits.c@{P\_\-bits.c}!bit_vector@{bit\_\-vector}}
\index{bit_vector@{bit\_\-vector}!P_bits.c@{P\_\-bits.c}}
\subsubsection{\setlength{\rightskip}{0pt plus 5cm}typedef struct {\bf \_\-bv}$\ast$ bit\_\-vector}\label{P__bits_8c_a0}




\subsection{Function Documentation}
\index{P_bits.c@{P\_\-bits.c}!bv_new@{bv\_\-new}}
\index{bv_new@{bv\_\-new}!P_bits.c@{P\_\-bits.c}}
\subsubsection{\setlength{\rightskip}{0pt plus 5cm}{\bf bit\_\-vector} bv\_\-new (bits)}\label{P__bits_8c_a6}


bit\_\-vector {\bf bv\_\-new}(bits) {\rm (p.\,\pageref{P__bits_8c_a6})} Create a new bit vector and set to zeros Reserve the memory space needed for the bit vector.

\begin{Desc}
\item[Parameters: ]\par
\begin{description}
\item[{\em 
bits}]size in bits of the vector \end{description}
\end{Desc}
\begin{Desc}
\item[Returns: ]\par
a new bit vector, or NULL if there is not enough memory \end{Desc}


\subsection{Variable Documentation}
\index{P_bits.c@{P\_\-bits.c}!bitnum@{bitnum}}
\index{bitnum@{bitnum}!P_bits.c@{P\_\-bits.c}}
\subsubsection{\setlength{\rightskip}{0pt plus 5cm}unsigned bitnum}\label{P__bits_8c_a2}


\begin{Desc}
\item[Parameters: ]\par
\begin{description}
\item[{\em 
N}]\'{u}mero de bit a fijar \end{description}
\end{Desc}
\index{P_bits.c@{P\_\-bits.c}!bv@{bv}}
\index{bv@{bv}!P_bits.c@{P\_\-bits.c}}
\subsubsection{\setlength{\rightskip}{0pt plus 5cm}struct {\bf \_\-bv}$\ast$ bv}\label{P__bits_8c_a5}


\index{P_bits.c@{P\_\-bits.c}!bv2@{bv2}}
\index{bv2@{bv2}!P_bits.c@{P\_\-bits.c}}
\subsubsection{\setlength{\rightskip}{0pt plus 5cm}struct {\bf \_\-bv} $\ast$ bv2}\label{P__bits_8c_a3}


\index{P_bits.c@{P\_\-bits.c}!bv3@{bv3}}
\index{bv3@{bv3}!P_bits.c@{P\_\-bits.c}}
\subsubsection{\setlength{\rightskip}{0pt plus 5cm}struct {\bf \_\-bv} $\ast$ bv3}\label{P__bits_8c_a4}


\index{P_bits.c@{P\_\-bits.c}!value@{value}}
\index{value@{value}!P_bits.c@{P\_\-bits.c}}
\subsubsection{\setlength{\rightskip}{0pt plus 5cm}char value}\label{P__bits_8c_a1}


